\documentclass[11pt]{article}
\usepackage[dvips]{epsfig}
\usepackage[dvips]{graphicx}
\usepackage{colordvi,color,colortbl}
\usepackage{amsmath}
\usepackage{amssymb,latexsym}
\usepackage[top=1.0cm,bottom=1cm,left=2cm,right=2cm,foot=.5cm]{geometry}




\begin{document}


\thispagestyle{empty}


\parbox[t]{1.\linewidth}{\centerline{\color{blue}\Large Molecular
    Dynamics and {\it ab initio} School 2023}}
\parbox[t]{1.\linewidth}{\hrulefill}


\vspace{10pt}

\centerline{\Huge{\color{red}{2D graphene}}}

\vspace{10pt}

{\bf{Introduction:}} For this project, you will investigate a 2
dimensions (2D) material which is Graphene (you may suggest another 2D
if you want for your project). We propose you to look at the
properties of different kind of defects within graphene: Potassium
adsorbed dopant, the so called 5-7 defect and a vacancy defect.\\


The preliminary work is mandatory as well as one of the more advanded
task as Potassium doping, 5-7 defect or vacancy defect.\\


{\bf{Preliminary work:}}

All the work proposed in this section may be done on a small computer
(laptop). You may want to reserve the use of the cluster for the
second part of this project.

At first your are going to look at the convergence of a calculation on
bulk material, {\it{i.e.}} you are going to consider the primitive cell
of graphene which is a 2D hexagonal cell. There is $2$ carbon atoms
within the primitive cell. You must be aware that graphene is a 2D
material and thus a vacuum gap has to be introduce in the out of plane
direction. As quantum espresso is using periodic conditions the
dimension of the unit cell in this direction must be chosen large
enough to ensure that the graphene sheet is not interacting with
itself.


The two main physical parameters which have to be tuned carefully are
the energy cut-off (mainly the plane wave cut-off) corresponding to
the parameter {\texttt{ecutwfc}} and the k-points sampling of the
Brillouin zone (parameter {\texttt{K\textunderscore
    POINTS}}). Basically, you will have to redo the same calculation
for increasing values of these parameters and look at the total energy
of the system which is the the key physical quantity. The convergence
is reached when the total energy is not varying anymore. Be aware that
you need to relax the cell (the lattice parameter may also be
monitored for the convergence) for each of your calculation (parameter
{\texttt{calculation = 'vc-relax'}}).\\

Once the optimal plane wave cut-off and k-points sampling have been
found, you may compute and plot the Band structure and the density of
states (DOS). Furthermore, you may check the dimension of the cell in
the direction perpendicular to the graphene plane to check that the
vaccum is large enough. The $c$ dimension of the cell is given by the
parameter {\texttt{celldm(3)}}. Its value in Bohr corresponds to
{\texttt{celldm(3)*celldm(1)}}. For this task, you should reduce the
cell dimension in this direction step by step and look at the total
energy and at the hydrostatic pressure and see, at which point, a
significant change is observed.\\


{\bf{Defect:}}

In this part you are going to introduce a defect in the graphene
sheet.  For this study you will need to consider a supercell. A
$4\times 4$ supercell is actually proposed for this project.  However,
if you have limited access to a supercomputer,you may consider to
define a smaller supercell in order to reduce the computational
time. Three different kinds of defect have been selected for you:

\begin{itemize}
\item
  Potassium atoms, when adsorbed on the graphene sheet, is leading to
  n type doping. A charge transfer between the Potassium atom and the
  graphene sheet is occurring leading to a shift of the Fermi level
  toward the conduction band. You may look at the projected density of
  states to investigate the localization effect induced by the dopant
  and also to quantify the charge transfer. Indeed, the charge
  tranfert (or charge carrier) is not an integer number of
  electron. To quantify the carrier density, it is mandatory to
  integrate the DOS from the top of the valence band (for graphene it
  is corresponds to the tip of the dirac cone) to the Fermi level. You
  will need to write your own code for that.\\

  
\item
  The 5-7 defect (also called Stone-Wales) corresponds to a rotation
  of 2 carbon atoms leading to the occurrence of 2 pentagons and 2
  heptagones. This is a very common defect within graphene. For this
  defect you may compute the DOS and the projected density of states
  (PDOS) to observe some localization effect for the atoms close to
  the defect.
  
\item
  Another common defect within graphene corresponds to vacancies. We
  propose to study the effect of a single vacancy within the graphene
  sheet. Single vacancy is not the most stable one (di-vacancy is
  actually more stable and its leading to a pentagon
  reconstruction). You may investigate this reconstruction and look at
  the variation of the bond lengths. you may also compute the DOS and
  the projected density of states (PDOS) to observe some localization
  effect for the atoms close to the defect.


  \vspace{5pt}

\end{itemize}


All endeavours/initiatives are strongly encouraged. If you need help,
you may request a zoom meeting with myself. Just email me
{\texttt{christophe.adessi@univ-lyon1.fr}} to make an appointment.


\end{document}
